%%%%%%%%%%%%%%%%%
% This is an example CV created using altacv.cls (v1.1.5, 1 December 2018) written by
% LianTze Lim (liantze@gmail.com), based on the
% Cv created by BusinessInsider at http://www.businessinsider.my/a-sample-resume-for-marissa-mayer-2016-7/?r=US&IR=T
%
%% It may be distributed and/or modified under the
%% conditions of the LaTeX Project Public License, either version 1.3
%% of this license or (at your option) any later version.
%% The latest version of this license is in
%%    http://www.latex-project.org/lppl.txt
%% and version 1.3 or later is part of all distributions of LaTeX
%% version 2003/12/01 or later.
%%%%%%%%%%%%%%%%

%% If you are using \orcid or academicons
%% icons, make sure you have the academicons
%% option here, and compile with XeLaTeX
%% or LuaLaTeX.
% \documentclass[10pt,a4paper,academicons]{altacv}

\PassOptionsToPackage{dvipsnames}{xcolor}

%% Use the "normalphoto" option if you want a normal photo instead of cropped to a circle
% \documentclass[10pt,a4paper,normalphoto]{altacv}

\documentclass[9pt,a4paper,ragged2e,normalphoto]{altacv}

%% AltaCV uses the fontawesome and academicon fonts
%% and packages.
%% See texdoc.net/pkg/fontawecome and http://texdoc.net/pkg/academicons for full list of symbols. You MUST compile with XeLaTeX or LuaLaTeX if you want to use academicons.

% Change the page layout if you need to
\geometry{left=0.5cm,right=11cm,marginparwidth=8.8cm,marginparsep=1.5cm,top=0.5cm,bottom=0.5cm}

% Change the font if you want to, depending on whether
% you're using pdflatex or xelatex/lualatex
\ifxetexorluatex
  % If using xelatex or lualatex:
  \setmainfont{Carlito}
\else
  % If using pdflatex:
  \usepackage[utf8]{inputenc}
  \usepackage[T1]{fontenc}
  \usepackage[default]{lato}
\fi

% Change the colours if you want to
\definecolor{Blue}{HTML}{4B85E1}
\definecolor{Black}{HTML}{000000}
\definecolor{Orange}{HTML}{0A1963}
\definecolor{LightGrey}{HTML}{666666}
\colorlet{heading}{Blue}
\colorlet{vide}{Black}
\colorlet{accent}{Blue}
\colorlet{profile}{Orange}
\colorlet{makecvheader}{Blue}
\colorlet{emphasis}{Orange}
\colorlet{body}{LightGrey}

% Change the bullets for itemize and rating marker
% for \cvskill if you want to
\renewcommand{\itemmarker}{{\small\textbullet}}
\renewcommand{\ratingmarker}{\faCircle}

\usepackage[colorlinks]{hyperref}

\begin{document}
\name{Mohamed CHERCHERI}
\tagline{Élève ingénieur à Télécom SudParis.}
% Cropped to square from https://en.wikipedia.org/wiki/Marissa_Mayer#/media/File:Marissa_Mayer_May_2014_(cropped).jpg, CC-BY 2.0
%\photo{2 cm}{meCv}
\personalinfo{
  \email{mohamed.chercheri@telecom-sudparis.eu}
  \phone{+33 06 24 54 28 47}
  \mailaddress{5, Clos de la Cathédrale, 91000 Evry}
  \location{Paris, France}
  \href{http://linkedin.com/in/chercheri-mohamed}{ \color{Orange}{\linkedin{chercheri-mohamed}}}
  \href{https://github.com/chercheri}{ \color{Orange}{\github{chercheri}}}
}

%% Make the header extend all the way to the right, if you want.
\begin{fullwidth}
\makecvheader
\end{fullwidth}

%% Depending on your tastes, you may want to make fonts of itemize environments slightly smaller
\AtBeginEnvironment{itemize}{\small}

%% Provide the file name containing the sidebar contents as an optional parameter to \cvsection.
%% You can always just use \marginpar{...} if you do
%% not need to align the top of the contents to any
%% \cvsection title in the "main" bar.
\cvsection[page1sidebar]{Profil}
\begin{quote}
``Actuellement élève ingénieur en
troisième année à Télécom SudParis spécialisé en Architecture de services répartis.

Étant travailleur, organisé et capable
de faire preuve d'autonomie, je suis à
la recherche d'un stage de fin d'études
de 6 mois (à partir de Février
2020).''
\end{quote}

\cvsection{Formation}

\cvevent{Cycle ingénieur}{Télécom SudParis, France}{Septembre 2018 -- 2020}{}
\begin{itemize}
\item Double-diplomation (Sup'Com, Tunisie) et Télécom SudParis (TSP)
\item Spécialisation en architecture de services informatiques répartis (ASR)
\end{itemize}
\divider

\cvevent{Cycle ingénieur}{École Supérieur des Communications de Tunis (Sup'Com), Tunisie}{Septembre 2016 -- Juin 2018}{}
\begin{itemize}
\item Spécialisation en système des télécoms (SYSTEL).
\end{itemize}
\divider

\cvevent{Cycle préparatoire}{Institut Préparatoire aux Études d'Ingénieurs El Manar (IPEIEM), Tunisie}{Septembre 2014 -- Juin 2016}{}
\begin{itemize}
\item Diplôme d'études universitaires de premier cycle en MP.
\end{itemize}
\divider

\cvevent{Baccalauréat}{Lycée les pères blancs, Tunisie}{juin 2014}{}
\begin{itemize}
\item Section: Mathématique - Moyenne: 15.96/20
\end{itemize}

\cvsection{Langues}
\cvskill{Français}{     DELF (B2) - VOLTAIRE}{5}
\cvskill{Arabe}{}{5}
\cvskill{Anglais}{      TOEIC (860/990)}{3}
\cvskill{Espagnol}{}{2}

%\cvsection{Life Philosophy}
%\begin{quote}
%``If you don't have any shadows, you're not standing in the %light.''
%\end{quote}

\cvsection{Réalisations}
\cvachievement{\faTrophy}{Résoudre une page de 100 problèmes}{Résolution d'une page de 100 problèmes de programmation sur la plateforme Codeforces }
\cvachievement{\faTrophy}{Étudier en France}{Avoir un rang 25/200 qui m'a offert une possibilité de double diplomation à Télécom Sud Paris}
\cvachievement{\faTrophy}{Étudier dans la deuxième meilleure école en Tunisie}{Un rang de 150/2300 élevés au concours national d’entrée aux cycles d’ingénieurs. }
\cvachievement{\faTrophy}{Champion de l'IPEIEM}{Vainqueur du championnat de football de l'institut préparatoire compétition de 32 équipes. }

\cvsection{CENTRES D’INTÉRÊT}

\cvtag{Cuisine}
\cvtag{Voyage} 
\cvtag{Football} 
\cvtag{Natation}
\cvtag{Programmation}

\clearpage

\end{document}
