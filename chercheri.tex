%%%%%%%%%%%%%%%%%
% This is an example CV created using altacv.cls (v1.1.5, 1 December 2018) written by
% LianTze Lim (liantze@gmail.com), based on the
% Cv created by BusinessInsider at http://www.businessinsider.my/a-sample-resume-for-marissa-mayer-2016-7/?r=US&IR=T
%
%% It may be distributed and/or modified under the
%% conditions of the LaTeX Project Public License, either version 1.3
%% of this license or (at your option) any later version.
%% The latest version of this license is in
%%    http://www.latex-project.org/lppl.txt
%% and version 1.3 or later is part of all distributions of LaTeX
%% version 2003/12/01 or later.
%%%%%%%%%%%%%%%%

%% If you are using \orcid or academicons
%% icons, make sure you have the academicons
%% option here, and compile with XeLaTeX
%% or LuaLaTeX.
% \documentclass[10pt,a4paper,academicons]{altacv}

%% Use the "normalphoto" option if you want a normal photo instead of cropped to a circle
% \documentclass[10pt,a4paper,normalphoto]{altacv}

\documentclass[9pt,a4paper,ragged2e,normalphoto]{altacv}

%% AltaCV uses the fontawesome and academicon fonts
%% and packages.
%% See texdoc.net/pkg/fontawecome and http://texdoc.net/pkg/academicons for full list of symbols. You MUST compile with XeLaTeX or LuaLaTeX if you want to use academicons.

% Change the page layout if you need to
\geometry{left=0.5cm,right=11cm,marginparwidth=8.8cm,marginparsep=1.5cm,top=0.5cm,bottom=0.5cm}

% Change the font if you want to, depending on whether
% you're using pdflatex or xelatex/lualatex
\ifxetexorluatex
  % If using xelatex or lualatex:
  \setmainfont{Carlito}
\else
  % If using pdflatex:
  \usepackage[utf8]{inputenc}
  \usepackage[T1]{fontenc}
  \usepackage[default]{lato}
\fi

% Change the colours if you want to
\definecolor{Blue}{HTML}{4B85E1}
\definecolor{Orange}{HTML}{0A1963}
\definecolor{LightGrey}{HTML}{666666}
\colorlet{heading}{Blue}
\colorlet{accent}{Blue}
\colorlet{emphasis}{Orange}
\colorlet{body}{LightGrey}

% Change the bullets for itemize and rating marker
% for \cvskill if you want to
\renewcommand{\itemmarker}{{\small\textbullet}}
\renewcommand{\ratingmarker}{\faCircle}


\begin{document}
\name{Mohamed CHERCHERI}
\tagline{Ingénieur informatique}
% Cropped to square from https://en.wikipedia.org/wiki/Marissa_Mayer#/media/File:Marissa_Mayer_May_2014_(cropped).jpg, CC-BY 2.0
\photo{2 cm}{DSC_0350}
\personalinfo{%
  % Not all of these are required!
  % You can add your own with \printinfo{symbol}{detail}
  \email{mohamed.chercheri@telecom-sudparis.eu}
  \phone{+33 06 24 54 28 47}
  \mailaddress{9, Rue Charles Fourier, 91000 Evry}
  \location{Paris, France}
  \linkedin{linkedin.com/in/chercheri-mohamed}
  \github{github.com/chercheri} % I'm just making this up though.
%   \orcid{orcid.org/0000-0000-0000-0000} % Obviously making this up too. If you want to use this field (and also other academicons symbols), add "academicons" option to \documentclass{altacv}
}

%% Make the header extend all the way to the right, if you want.
\begin{fullwidth}
\makecvheader
\end{fullwidth}

%% Depending on your tastes, you may want to make fonts of itemize environments slightly smaller
\AtBeginEnvironment{itemize}{\small}

%% Provide the file name containing the sidebar contents as an optional parameter to \cvsection.
%% You can always just use \marginpar{...} if you do
%% not need to align the top of the contents to any
%% \cvsection title in the "main" bar.
\cvsection[page1sidebar]{Expérience Professionnelle}

\cvevent{Stagiaire : Stage de formation humaine}{Tunisie Télécom}{Juillet 2017}{Tunis, Tunisie}
\begin{itemize}
\item Premier pas dans le monde professionnel.
\item Participation à la réparation des lignes téléphoniques des abonnés.
\end{itemize}

\cvsection{Expérience académique}

\cvevent{Projet Cassiopée}{}{Janvier -- Juillet 2019}{Télécom SudParis}
\begin{itemize}
\item Développement d'une plateforme B2B.
\item Technologies : PHP,HTML,CSS,React.js,SQLite.
\end{itemize}

\divider

\cvevent{Projet web}{}{septembre -- Décembre 2018}{Télécom SudParis}
\begin{itemize}
\item Développement d'un site web pour une agence de voyage qui fournit des offres aux clients.
\item Technologies : PHP,HTML,CSS,Twig,SQLite.
\end{itemize}

\divider

\cvevent{Projet PISTE}{}{janvier -- Juin 2018}{Sup'Com, Tunisie}
\begin{itemize}
\item Solution IOT de perception multi-capteurs.
\item Suivie d'une personne dépendante à travers des capteurs et une interface web.    
\item Technologies : Angular,NodeJS,MongoDB.
\end{itemize}

\divider

\cvevent{Projet de développement mobile}{}{Septembre -- Décembre 2017}{Sup'Com, Tunisie}
\begin{itemize}
\item Application mobile qui facilite le suivi parental de la vie scolaire des enfants.    
\item Technologies : JAVA,XML,MySQL en utilisant AndroidStudio.
\end{itemize}

\cvsection{Compétences Techniques}

\cvtag{C/C++} 
\cvtag{Java/J2EE}
\cvtag{PHP} 
\cvtag{JavaScript}
\cvtag{HTML}
\cvtag{CSS}

%\divider\smallskip

\cvtag{Symfony}
\cvtag{Spring}
\cvtag{Angular}
\cvtag{React.js}
\cvtag{Maven}
\cvtag{Hibernate}
%\divider\smallskip
\cvtag{MySQL}

%\divider\smallskip
\cvtag{Linux/Ubuntu}
\cvtag{Windows}

\cvsection{Extra Scolaire}

\cvevent{Membre du comité logistique du forum}{}{Janvier 2017 -- Juin 2018}{Sup'Com}
\begin{itemize}
\item Participation, avec un groupe d’étudiants, à la réalisation du forum
annuel et de la journée portes ouvertes.
\end{itemize}

\divider

\cvevent{Club: Sup'Story}{}{Novembre 2016 -- Juin 2018}{Sup'Com}
\begin{itemize}
\item Fondateur du club Sup'Story qui a pour but la valorisation du patrimoine de Sup'Com.
\item Collecte de la mémoire de Sup'Com à travers les photos, les interviews.
\end{itemize}


\clearpage

\end{document}
